
%== Iniciar Marca D'água (Water Mark)
%== Necessário: "\usepackage{eso-pic}"
%== Necessário: configurar "BackgorundPic"
\AddToShipoutPicture{\BackgroundPic} 


%== Gerar capa
\imprimircapa


%== Folha de rosto
%== O * indica que haverá a ficha bibliográfica.
% \imprimirfolhaderosto*
\imprimirfolhaderosto


%== Inserir Prefácio (ambiente 'resumo')
\begin{resumo}[Prefácio]
A Física é uma Ciência que exerce profunda influência em muitas outras áreas do saber, tais como a Engenharia, a Biologia, a Química, etc. Estudar os fenômenos físicos é de vital importância para a compreensão dos princípios que regem o mundo e desenvolver a habilidade de trabalhar com eles, aplicando-os para tornar o abstrato em concreto: é a essência da tecnologia. 
Planejamos esta apostila como objeto de apoio ao estudante, visando o ensino mais eficiente, produtivo, interessante e agradável.
Temos, contudo, a ciência de que sempre podemos melhorar, mas, para isso, precisamos da colaboração de cada estudante que venha usufruir deste curso, concedendo-nos suas críticas e sugestões.
\vspace{1cm}                            
\hfill Os professores.
\end{resumo}


%== Inserir lista de ilustrações.
\pdfbookmark[0]{\listfigurename}{lof}
\listoffigures*
\cleardoublepage


%== Inserir lista de tabelas.
\pdfbookmark[0]{\listtablename}{lot}
\listoftables*
\cleardoublepage


%== Inserir o sumario.
\pdfbookmark[0]{\contentsname}{toc}
\tableofcontents*
\cleardoublepage


%== Elementos Textuais.
\textual

%Fim
